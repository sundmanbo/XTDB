\documentclass[preprint,review,12pt]{elsarticle}

%% Use the option review to obtain double line spacing
%% \documentclass[authoryear,preprint,review,12pt]{elsarticle}

% needed to have pdf files 
\pdfsuppresswarningpagegroup=1

%% Use the options 1p,twocolumn; 3p; 3p,twocolumn; 5p; or 5p,twocolumn
%% for a journal layout:
%% \documentclass[final,1p,times]{elsarticle}
%% \documentclass[final,1p,times,twocolumn]{elsarticle}
%% \documentclass[final,3p,times]{elsarticle}
%% \documentclass[final,3p,times,twocolumn]{elsarticle}
%% \documentclass[final,5p,times]{elsarticle}
%% \documentclass[final,5p,times,twocolumn]{elsarticle}

%% For including figures, graphicx.sty has been loaded in
%% elsarticle.cls. If you prefer to use the old commands
%% please give \usepackage{epsfig}

%% The amssymb package provides various useful mathematical symbols
\usepackage{amssymb}
%% The amsthm package provides extended theorem environments
%% \usepackage{amsthm}

%% The lineno packages adds line numbers. Start line numbering with
%% \begin{linenumbers}, end it with \end{linenumbers}. Or switch it on
%% for the whole article with \linenumbers.
%% \usepackage{lineno}

%---------------------------------------------- begin bosse special
\usepackage{xcolor}
\usepackage[normalem]{ulem}
% end of XML tag
\newcommand\eoxml{/\hspace{-4pt}>}

\usepackage{xcolor}
\usepackage{ulem}
\usepackage{framed}

%footnote using capital letter
%\renewcommand{\thefootnote}{\Alph{footnote}}
%\renewcommand{\thefootnote}{{\arabic{footnote}}

% For appendices
%\usepackage[titletoc,title,header]{appendix}
%------------------------------------------------- end bosse special

\journal{Calphad}

\usepackage{graphicx,subfigure}              % with figures
\usepackage[titletoc,title,header]{appendix}

\begin{document}

%% \begin{frontmatter}

%% Title, authors and addresses

%% use the tnoteref command within \title for footnotes;
%% use the tnotetext command for theassociated footnote;
%% use the fnref command within \author or \address for footnotes;
%% use the fntext command for theassociated footnote;
%% use the core command within \author for corresponding author footnotes;
%% use the cortext command for theassociated footnote;
%% use the ead command for the email address,
%% and the form \ead[url] for the home page:
%% \title{Title\tnoteref{label1}}
%% \tnotetext[label1]{}
%% \author{Name\corref{cor1}\fnref{label2}}
%% \ead{email address}
%% \ead[url]{home page}
%% \fntext[label2]{}
%% \cortext[cor1]{}
%% \address{Address\fnref{label3}}
%% \fntext[label3]{}

\title{XTDB, an XML based format for Calphad databases}

%% use optional labels to link authors explicitly to addresses:
%% \author[label1,label2]{}
%% \address[label1]{}
%% \address[label2]{}

% 1 bo.sundman@GMAIL.COM
% 2 ABE.Taichi@NIMS.GO.JP
% 3 avdw@BROWN.EDU                  Axel van de Walle
% 4 b.hallstedt@IWM.RWTH-AACHEN.DE
% 5 erwin.povoden-karadeniz@TUWIEN.AC.AT
% 6 aurelie.jacob@tuwien.ac.at 
% 7 fabio.miani@UNIUD.IT
% 8 fan.zhang@COMPUTHERM.COM
% 9 shuanglin.chen@COMPUTHERM.COM
% 10 lina@THERMOCALC.SE            Lina Kjellqvist
% 11 reza@THERMOCALC.SE            Reza Naraghi
% 12 malin@KTH.SE                  Malin Selleby
% 13 nathdupin@CTHERMO.FR          Nathalie Dupin
% 14 richard.otis@OUTLOOK.COM
% 15 urkattner@GMAIL.COM           Ursula R Kattner
% 16 ft@gtt-technologies.de        Florian Tang
% 17 alexander.pisch@simap.grenoble-inp.fr 

  \author[bosse]{Bo Sundman}       % 1
  \ead{bo.sundman@gmail.com}
  \author[fabio]{Fabio Miani}      % 2 fabio.miani@uniud.it
  \author[alex]{Axel van de Walle} % 3 avdw@brown.edu
  \author[bengt]{Bengt Hallstedt}  % 4 b.hallstedt@iwm.rwth-aachen.de
  \author[urk]{Ursula R Kattner}   % 15 urkattner@gmail.com
  \author[gtt]{Florian Tang}       % 16 ft@gtt-technologies.de
  \author[abe]{Taichi Abe}         % 2 abe.taichi@nims.go.jp
  \author[tcsab]{Reza Naraghi}      % 11 reza@thermocalc.se
  \author[tuwien]{Erwin Povoden-Karadeniz} %erwin.povoden-karadeniz@tuwien.ac.at
  \author[tuwien]{Aurelie Jacob}  % 6 aurelie.jacob@tuwien.ac.at 
  \author[computherm]{Shuanglin Chen} % 9 shuanglin.chen@computherm.com
%  \author[computherm]{Fan Zhang}      % 8 fan.zhang@computherm.com
%  \author[tcsab]{Lina Kjellqvist}   % 10 lina@thermocalc.se
%  \author[nath]{Nathalie Dupin}    % 13 nathdupin@cthermo.fr
  \author[richard]{Richard Otis}   % 14 richard.otis@outlook.com
  \author[shobu]{Kazuhisa Shobu}   % 13 Catcalc k,shobu@rictsystems.com
  \author[kth]{Malin Selleby}      % 12 malin@kth.se     
  \author[simap]{Alexander Pisch} % alexander.pisch@simap.grenoble-inp.fr

  \address[bosse]{OpenCalphad, 9 All{\'e}e de l'Acerma, 91190 Gif sur
    Yvette, France}
  \address[fabio]{DPIA, University of Udine, Via delle Scienze 208, 33100 Udine, Italy}
  \address[alex]{School of Engineering, Brown University, Providence, RI 02912, USA}
  \address[bengt]{IWM, RWTH Aachen University, Augustinerbach 4, 52062
    Aachen, Germany}
  \address[urk]{MSE, NIST, 100 Bureau Drive, Stop 8555, Gaithersburg, MD 20899, USA}
  \address[gtt]{GTT-Technologies, 52134 Herzogenrath, Germany}
  \address[abe]{NIMS, Sengen, Tsukuba, Japan}
  \address[tcsab]{Thermo-Calc Software AB,
    R{\aa}sundav{\"a}gen 18, 169 67 Solna, Sweden}
  \address[tuwien]{TU Wien, Getreidemarkt 9, 1060
    Vienna, Austria}
  \address[computherm]{CompuTherm, Yellowstone Dr.,
    Madison, WI 53719, USA}
%  \address[nath]{Calcul Thermodynamique, 63670 Orcet, France}
%  \address[richard]{Proteus Space Inc, Los Angeles, CA 90021, USA}
  \address[richard]{California Institute of Technology, Pasadena, CA 91109 USA}
  \address[shobu]{RICT, Inc. 674-18, Tashiro-hoka, Tosu, Saga, 841-0016, Japan.}
  \address[kth]{MSE, KTH Royal Institute of Technology,
    Brinellv{\"a}gen 23, 100 44 Stockholm, Sweden}
  \address[simap]{SIMAP, 1130, Rue de la Piscine, 38402 Saint-Martin
    d'H{\`e}res, France}

{\large \bf As sent for publication}

\begin{abstract}
  The Calphad method uses models which depend on assessed parameters
  to describe the thermodynamic properties of materials.  These model
  parameters are assessed by researchers and students using
  experimental and theoretical data on binary and ternary systems
  which can be merged to multicomponent databases and used to
  calculate properties and simulate processes for a wide range of
  materials.

  There are several different software using the Calphad method for
  calculations and they can use slightly different models and database
  formats.  This paper will give a short background to the current
  state of database development and proposes a new format based on the
  eXtensive Markup Language (XML) as a unified database format.  This
  change is particularly important as several new models for the pure
  elements are currently introduced in the Calphad databases.

%  Some features or peculiarities of the current databases are also discussed.

\end{abstract}


\newpage

\begin{appendix}

\setcounter{equation}{0}
\renewcommand{\theequation}{C\arabic{equation}}
\setcounter{figure}{0}
\renewcommand{\thefigure}{C\arabic{figure}}

\section{Definition of XTDB tags}

%%%%%%%%%%%%%%%%%%%%%%%%%%%%%%%%%%%%%%%%%%%%%%%%%%%%%%%%%%%%%%%%%%%%%%
% Appendix B Examples of an XTDB file
%%%%%%%%%%%%%%%%%%%%%%%%%%%%%%%%%%%%%%%%%%%%%%%%%%%%%%%%%%%%%%%%%%%%%%

\setcounter{equation}{0}
\renewcommand{\theequation}{B\arabic{equation}}
\setcounter{figure}{0}
\renewcommand{\thefigure}{B\arabic{figure}}

\section{Examples of XTDB files}\label{sc:examples}

\subsection{A complete Al-C database}\label{sc:alcexample}

{\small
\begin{verbatim}
<XTDB version="0.1.5">
<Defaults LowT="10" HighT="6000" Elements="VA /-" />
<Element Id="AL" Refstate="FCC_A1" Mass="26.982" H298="4577.3" S298="28.322" />
<Element Id="C" Refstate="GRAPHITE" Mass="12.011" H298="1054" S298="5.7423" />
<Species Id="VA" Stoichiometry="VA" />
<Species Id="AL" Stoichiometry="AL" />
<Species Id="C" Stoichiometry="C" />
<Phase Id="LIQUID" Configuration="CEF" State="L" >
  <Sublattices NumberOf="1" Multiplicities="1" >
    <Constituents Sublattice="1" List="AL C" />
  </Sublattices>
 <AmendPhase Models="LIQ2STATE" />
</Phase>
<Phase Id="AL4C3" Configuration="CEF" State="S" >
  <Crystallography PearsonSymbol="hR21" SpaceGroup="166" Prototype="Al4C3" />
  <Sublattices NumberOf="2" Multiplicities="4 3" >
    <Constituents Sublattice="1" List="AL" />
    <Constituents Sublattice="2" List="C" />
  </Sublattices>
  <AmendPhase Models="GEIN" />
</Phase>
<Phase Id="BCC_A2" Configuration="CEF" State="S" >
  <Crystallography Structurbericht="A2" PearsonSymbol="cI2" Prototype="W" />
  <Sublattices NumberOf="2" Multiplicities="1 3" >
    <Constituents Sublattice="1" List="AL" />
    <Constituents Sublattice="2" List="C VA" />
  </Sublattices>
  <AmendPhase Models="GEIN" />
</Phase>
<Phase Id="DIAMOND" Configuration="CEF" State="S" >
  <Crystallography StructurBerict="A4" PearsonSymbol="cF8" Prototype="C" />
  <Sublattices NumberOf="1" Multiplicities="1" >
    <Constituents Sublattice="1" List="C" />
  </Sublattices>
  <AmendPhase Models="GEIN" />
</Phase>
<Phase Id="FCC_A1" Configuration="CEF" State="S" >
  <Crystallography StructurBericht="A1" PearsonSymbol="cF4" Prototype="Cu" />
  <Sublattices NumberOf="2" Multiplicities="1 1" >
    <Constituents Sublattice="1" List="AL" />
    <Constituents Sublattice="2" List="C VA" />
  </Sublattices>
  <AmendPhase Models="GEIN" />
</Phase>
<Phase Id="GRAPHITE" Configuration="CEF" State="S" >
  <Crystallography StructurBericht="A9" PearsonSymbol="hP4"  Prototype="C" />
  <Sublattices NumberOf="1" Multiplicities="1" >
    <Constituents Sublattice="1" List="C" />
  </Sublattices>
  <AmendPhase Models="GEIN" />
</Phase>
<Phase Id="HCP_A3" Configuration="CEF" State="S" >
  <Crystallography StructurBericht="A3" PearsonSymbol="hP2" Prototype="Mg" />
  <Sublattices NumberOf="2" Multiplicities="1 0.5" >
    <Constituents Sublattice="1" List="AL" />
    <Constituents Sublattice="2" List="C VA" />
  </Sublattices>
  <AmendPhase Models="GEIN" />
</Phase>
<TPfun Id="R"     Expr="8.31451;" />
<TPfun Id="RTLNP" Expr="R*T*LN(1.0E-5)*P);" />
<TPfun Id="G0AL4C3" Expr=" -277339-.005423368*T**2;" /> 
<TPfun Id="GTSERAL" Expr=" -.001478307*T**2-7.83339395E-07*T**3;" /> 
<TPfun Id="GTSERCC" Expr=" -.00029531332*T**2-3.3998492E-16*T**5;" /> 
<TPfun Id="G0BCCAL" Expr=" +GHSERAL+10083;" /> 
<TPfun Id="G0HCPAL" Expr=" +GHSERAL+5481;" /> 
<TPfun Id="GHSERAL" Expr=" -8160+GTSERAL;" /> 
<TPfun Id="GHSERCC" Expr=" -17752.213+GEGRACC+GTSERCC;" /> 
<TPfun Id="G0DIACC" Expr=" -16275.202-9.1299452E-05*T**2-2.1653414E-16*T**5;" /> 
<TPfun Id="GEDIACC" Expr=" +0.2318*GEIN(+813.6)+.01148*GEIN(+345.4)
       -0.236743*GEIN(+1601.4);" /> 
<TPfun Id="G0LIQAL" Expr=" -209-3.777*T-.00045*T**2;" /> 
<TPfun Id="G0LIQCC" Expr=" +63887-8.2*T-.0004185*T**2;" /> 
<TPfun Id="GEGRACC" Expr=" -0.5159523*GEIN(+1953.3)+0.121519*GEIN(+448)
       +0.3496843*GEIN(+947)+.0388463*GEIN(+192.7)+.005840323*GEIN(+64.5);" /> 
<Parameter Id="G(LIQUID,AL;0)" Expr=" +G0LIQAL;" Bibref="21HE" />
<Parameter Id="LNTH(LIQUID,AL;0)" Expr=" +LN(+254);" Bibref="21HE" />
<Parameter Id="GD(LIQUID,AL;0)" Expr=" +13398-R*T-0.16597*T*LN(+T);" Bibref="21HE" />
<Parameter Id="G(LIQUID,C;0)" Expr=" +G0LIQCC;" Bibref="21HE" />
<Parameter Id="LNTH(LIQUID,C;0)" Expr=" +LN(+1400);" Bibref="21HE" />
<Parameter Id="GD(LIQUID,C;0)" Expr=" +59147-49.61*T+2.9806*T*LN(+T);" Bibref="21HE" />
<Parameter Id="G(LIQUID,AL,C;0)" Expr=" +20994-22*T;" Bibref="21HE" />
<Parameter Id="G(AL4C3,AL:C;0)" Expr=" +G0AL4C3-3.08*GEIN(+401)+3.08*GEIN(+1077);" Bibref="21HE" />
<Parameter Id="LNTH(AL4C3,AL:C;0)" Expr=" +LN(+401);" Bibref="21HE" />
<Parameter Id="G(BCC_A2,AL:C;0)" Expr=" +GTSERAL+3*GTSERCC+1006844;" Bibref="21HE" />
<Parameter Id="LNTH(BCC_A2,AL:C;0)" Expr=" +LN(+863);" Bibref="21HE" />
<Parameter Id="G(BCC_A2,AL:VA;0)" Expr=" +G0BCCAL;" Bibref="21HE" />
<Parameter Id="LNTH(BCC_A2,AL:VA;0)" Expr=" +LN(+233);" Bibref="21HE" />
<Parameter Id="G(BCC_A2,AL:C,VA;0)" Expr=" -819896+14*T;" Bibref="21HE" />
<Parameter Id="G(DIAMOND,C;0)" Expr=" +G0DIACC+GEDIACC;" Bibref="21HE" />
<Parameter Id="LNTH(DIAMOND,C;0)" Expr=" +LN(+1601.4);" Bibref="21HE" />
<Parameter Id="G(FCC_A1,AL:C;0)" Expr=" +GTSERAL+GTSERCC+57338;" Bibref="21HE" />
<Parameter Id="LNTH(FCC_A1,AL:C;0)" Expr=" +LN(+549);" Bibref="21HE" />
<Parameter Id="G(FCC_A1,AL:VA;0)" Expr=" +GHSERAL;" Bibref="21HE" />
<Parameter Id="LNTH(FCC_A1,AL:VA;0)" Expr=" +LN(+283);" Bibref="21HE" />
<Parameter Id="G(FCC_A1,AL:C,VA;0)" Expr=" -70345;" Bibref="21HE" />
<Parameter Id="G(GRAPHITE,C;0)" Expr=" +GHSERCC;" Bibref="21HE" />
<Parameter Id="LNTH(GRAPHITE,C;0)" Expr=" +LN(+1953.3);" Bibref="21HE" />
<Parameter Id="G(HCP_A3,AL:C;0)" Expr=" +GTSERAL+0.5*GTSERCC+2176775;" Bibref="21HE" />
<Parameter Id="LNTH(HCP_A3,AL:C;0)" Expr=" +LN(+452);" Bibref="21HE" />
<Parameter Id="G(HCP_A3,AL:VA;0)" Expr=" +G0HCPAL;" Bibref="21HE" />
<Parameter Id="LNTH(HCP_A3,AL:VA;0)" Expr=" +LN(+263);" Bibref="21HE" />
<Parameter Id="G(HCP_A3,AL:C,VA;0)" Expr=" 0;" Bibref="21HE" />
<Bibliography>
<Bibitem Id="21HE" Text="Z. He, B. Kaplan, H. Mao, M. Selleby, Calphad (2021) 102250" /> 
</Bibliography>
</XTDB>
\end{verbatim}
}

\subsection{A $\sigma$ phase with EBEF and DisorderedPart}\label{sc:sigma-ebef}

{\small
\begin{verbatim}
<Element Id="AL" Refstate="FCC_A1" Mass="26.982" H298="4577.3" S298="28.322" />
<Element Id="Cr" Refstate="BCC_A2" Mass="51.996" H298="4050" S298="23.56" />
<Element Id="FE" Refstate="BCC_A2" Mass="55.847" H298="4489" S298="27.28" />

<Phase Id="SIGMA" Configuration="CEF" State="S" >
  <Crystallography StructurBericht="D8_b" PearsonSymbol="tP30" SpaceGroup="P4_2/mnm" />
  <Sublattices NumberOf="5" Multiplicities="2 4 8 8 8" >
    <Constituents Sublattice="1" List="AL CR FE" />
    <Constituents Sublattice="2" List="AL CR FE" />
    <Constituents Sublattice="3" List="AL CR FE" />
    <Constituents Sublattice="4" List="AL CR FE" />
    <Constituents Sublattice="5" List="AL CR FE" />
  </Sublattices>
  <AmendPhase > <!-- EBEF is used for the parameters -->
    <DisorderedPart Sum="5" />
  </AmendPhase>
</Phase>

<!-- Endmember parameters are in the disordered part. They are for a single atom and
     should be multiplied by 30 (the sum of the multiplicities) by the software. -->

<Parameter Id="G(SIGMA,AL;0)" Expr=" +GSIGMA_AL;" Bibref="SGTE2025"/>
<Parameter Id="G(SIGMA,CR;0)" Expr=" +GSIGMA_CR;" Bibref="SGTE2025"/>
<Parameter Id="G(SIGMA,FE;0)" Expr=" +GSIGMA_FE;" Bibref="SGTE2025"/>

<!-- Below the 20 EBEF excess endmember parameters for the ordered part of Al-Cr.  
     Without wildcards there are 32 endmembers.  The * can represent any element -->

<Parameter Id="G(SIGMA,AL:CR:*:*:*;0)" Expr=" SIGMA_X_AL1CR2;" />
<Parameter Id="G(SIGMA,AL:*:CR:*:*;0)" Expr=" SIGMA_X_AL1CR3;" />
<Parameter Id="G(SIGMA,AL:*:*:CR:*;0)" Expr=" SIGMA_X_AL1CR4;" />
<Parameter Id="G(SIGMA,AL:*:*:*:CR;0)" Expr=" SIGMA_X_AL1CR5;" />
<Parameter Id="G(SIGMA,*:AL:CR:*:*;0)" Expr=" SIGMA_X_AL2CR3;" />
<Parameter Id="G(SIGMA,*:AL:*:CR:*;0)" Expr=" SIGMA_X_AL2CR4;" />
<Parameter Id="G(SIGMA,*:AL:*:*:CR;0)" Expr=" SIGMA_X_AL2CR5;" />
<Parameter Id="G(SIGMA,*:*:AL:CR:*;0)" Expr=" SIGMA_X_AL3CR4;" />
<Parameter Id="G(SIGMA,*:*:AL:*:CR;0)" Expr=" SIGMA_X_AL3CR5;" />
<Parameter Id="G(SIGMA,*:*:*:AL:CR;0)" Expr=" SIGMA_X_AL4CR5;" />
<Parameter Id="G(SIGMA,CR:AL:*:*:*;0)" Expr=" SIGMA_X_CR1AL2;" />
<Parameter Id="G(SIGMA,CR:*:AL:*:*;0)" Expr=" SIGMA_X_CR1AL3;" />
<Parameter Id="G(SIGMA,CR:*:*:AL:*;0)" Expr=" SIGMA_X_CR1AL4;" />
<Parameter Id="G(SIGMA,CR:*:*:*:AL;0)" Expr=" SIGMA_X_CR1AL5;" />
<Parameter Id="G(SIGMA,*:CR:AL:*:*;0)" Expr=" SIGMA_X_CR2AL3;" />
<Parameter Id="G(SIGMA,*:CR:*:AL:*;0)" Expr=" SIGMA_X_CR2AL4;" />
<Parameter Id="G(SIGMA,*:CR:*:*:AL;0)" Expr=" SIGMA_X_CR2AL5;" />
<Parameter Id="G(SIGMA,*:*:CR:AL:*;0)" Expr=" SIGMA_X_CR3AL4;" />
<Parameter Id="G(SIGMA,*:*:CR:*:AL;0)" Expr=" SIGMA_X_CR3AL5;" />
<Parameter Id="G(SIGMA,*:*:*:CR:AL;0)" Expr=" SIGMA_X_CR4AL5;" />

<!-- There are also 20 EBEF endmember parameters for Al-Fe and Cr-Fe -->

<Parameter Id="G(SIGMA,AL:FE:*:*:*;0)" Expr=" SIGMA_X_AL1FE2;" />

<Parameter Id="G(SIGMA,CR:FE:*:*:*;0)" Expr=" SIGMA_X_CR1FE2;" />

\end{verbatim}
}

The {\bf TPfun}s SIGMA\_X\_AsBt can be fitted to DFT calculated endmembers.
In a ternary EBEF there are 63 parameters, without wildcards there are
$3^5 = 243$ endmembers.

\subsection{A $\sigma$ phase with EBEF and DisorderedPart}\label{sc:sigma-ebef2}

Below is a suggestion of a future possibility for a shorter notation.

{\small
\begin{verbatim}
<Element Id="AL" Refstate="FCC_A1" Mass="26.982" H298="4577.3" S298="28.322" />
<Element Id="Cr" Refstate="BCC_A2" Mass="51.996" H298="4050" S298="23.56" />
<Element Id="FE" Refstate="BCC_A2" Mass="55.847" H298="4489" S298="27.28" />

<Phase Id="SIGMA" Configuration="CEF" State="S" >
  <Crystallography StructurBericht="D8_b" PearsonSymbol="tP30" SpaceGroup="P4_2/mnm" />
  <Sublattices NumberOf="5" Multiplicities="2 4 8 8 8" >
    <Constituents Sublattice="1" Wyckoff="2a" List="AL CR FE" />
    <Constituents Sublattice="2" Wyckoff="4f" List="AL CR FE" />
    <Constituents Sublattice="3" Wyckoff="8i1" List="AL CR FE" />
    <Constituents Sublattice="4" Wyckoff="8i2" List="AL CR FE" />
    <Constituents Sublattice="5" Wyckoff="8j" List="AL CR FE" />
  </Sublattices>
  <AmendPhase > Models="EBEF"  <!-- EBEF notation is used for the parameters -->
    <DisorderedPart Sum="5" />
  </AmendPhase>
</Phase>

<!-- Endmember parameters in the disordered part as in~\ref{sc:sigma-ebef.}
     The notation below use @ character indicate the sublattice of the constituent
     in the ordered part.  -->

<Parameter Id="G(SIGMA,AL@1:CR@2)" Expr=" SIGMA_X_AL1CR2;" />
<Parameter Id="G(SIGMA,AL@1:CR@3)" Expr=" SIGMA_X_AL1CR3;" />
<Parameter Id="G(SIGMA,AL@1:CR@4)" Expr=" SIGMA_X_AL1CR4;" />
<Parameter Id="G(SIGMA,AL@1:CR@5)" Expr=" SIGMA_X_AL1CR5;" />
<Parameter Id="G(SIGMA,AL@2:CR@3)" Expr=" SIGMA_X_AL2CR3;" />
<Parameter Id="G(SIGMA,AL@2:CR@4)" Expr=" SIGMA_X_AL2CR4;" />

\end{verbatim}
}

\subsection{An FCC phase with wildcards and DisorderedPart}\label{sc:fcc-2part}

{\small
\begin{verbatim}
<Element Id="AL" Refstate="FCC_A1" Mass="26.982" H298="4577.3" S298="28.322" />
<Element Id="Cr" Refstate="BCC_A2" Mass="51.996" H298="4050" S298="23.56" />
<Element Id="FE" Refstate="BCC_A2" Mass="55.847" H298="4489" S298="27.28" />
<Element Id="C"  Refstate="GRAPHITE" Mass="12.011" H298="1054" S298="5.7423" />

<Phase Id="FCC_4SL" Configuration="CEF" State="S" >
  <Sublattices NumberOf="5" Multiplicities="0.25 0.25 0.25 0.25 1" >
    <Constituents Sublattice="1" List="AL CR FE" />
    <Constituents Sublattice="2" List="AL CR FE" />
    <Constituents Sublattice="3" List="AL CR FE" />
    <Constituents Sublattice="4" List="AL CR FE" />
    <Constituents Sublattice="5" List="Va C" />
  </Sublattices>
  <AmendPhase Models="IHJREST GEIN FCC4PERM" >
    <DisorderedPart Sum="4" Subtract="N" />
  </AmendPhase>
</Phase>

<!-- The first 4 sublattices are for L1_2 and L1_0 ordering.
     Endmember parameters in the disordered part with no ordering.
     There can also be excess parameters to describe the disordered state.
     The disordered parameters are for the same phase, but have fewer sublattices. -->
<Parameter Id="G(FCC_4SL,AL:VA;0)" Expr=" +GHSERAL;" Bibref="21HE" />
<Parameter Id="LNTH(FCC_4SL,AL:VA;0)" Expr=" +LN(+283);" Bibref="21HE" />
<Parameter Id="G(FCC_4SL,AL:C;0)" Expr=" +GTSERAL+GTSERCC+57338;" Bibref="21HE" />
<Parameter Id="LNTH(FCC_4SL,AL:C;0)" Expr=" +LN(+549);" Bibref="21HE" />
<Parameter Id="G(FCC_4SL,CR:VA;0)" Expr=" +GFCC_CR;" Bibref="SGTE2025"/>
<Parameter Id="G(FCC_4SL,FE:VA;0)" Expr=" +GFCC_FE;" Bibref="SGTE2025"/>

<!-- Some excess parameters to descrive the stable disordered phase -->
<Parameter Id="G(FCC_4SL,AL:C,VA;0)" Expr=" -70345;" Bibref="21HE" />
<Parameter Id="G(FCC_4SL,AL,CR:VA;0)" Expr=" GFCC_X_ALCR0;" />
<Parameter Id="G(FCC_4SL,AL,CR:VA;1)" Expr=" GFCC_X_ALCR1;" />

<!-- Some examples of parameters in the ordered part of the FCC phase.
     A parameter G(FCC_4SL,AL:AL:AL:CR:VA;0) is permuted 4 times.
     A parameter G(FCC_4SL,AL:AL:CR:CR:VA;0) is permuted 6 times.  -->
<Parameter Id="G(FCC_4SL,AL:AL:AL:CR:VA;0)" Expr=" GFCC_AL3CR1;" />
<Parameter Id="G(FCC_4SL,AL:AL:CR:CR:VA;0)" Expr=" GFCC_AL2CR2;" />
<Parameter Id="G(FCC_4SL,AL:CR:CR:CR:VA;0)" Expr=" GFCC_AL1CR3;" />
<Parameter Id="G(FCC_4SL,AL:AL:AL:FE:VA;0)" Expr=" GFCC_AL3FE1;" />

<!-- An excess parameters with wildcards using the assumption that the AL-CR
     interaction is independent of the constituents on the other sublattices.
     This parameter is also permuted 4 times. -->
<Parameter Id="G(FCC_4SL,AL,CR:*:*:*:VA;0)" Expr=" GFCC_XO_ALCR;" />

<!-- This parameter approximate SRO both in ordered and disordered,
     it is permuted 6 times.  -->
<Parameter Id="G(FCC_4SL,AL,CR:AL,CR:*:*:VA;0)" Expr=" GFCC_SRO_ALCR;" />

\end{verbatim}
}


\end{appendix}

\end{document}

