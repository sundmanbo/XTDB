\documentclass[10pt]{article}

\usepackage[utf8]{inputenc}
\usepackage{graphicx,subfigure}
\usepackage{amssymb}
\usepackage{makeidx}
\usepackage[normalem]{ulem}
\makeindex
\topmargin 1mm
\oddsidemargin 1mm
\evensidemargin 1mm
\textwidth 160mm
\textheight 210mm
\parskip 2mm
\parindent 3mm
%
%\usepackage[firstpage]{draftwatermark}
%\SetWatermarkScale{4}
%\setcounter{tocdepth}{1}
%\setcounter{secnumdepth}{4}
%\pagestyle{empty}


\begin{document}

%%%%%%%%%%%%%%%%%%%%%%%%%%%%%%%%%%%%%%%%%%%%%%%%%%%%%%%%%%%%%%%%%%%%%%
% Appendix A
%%%%%%%%%%%%%%%%%%%%%%%%%%%%%%%%%%%%%%%%%%%%%%%%%%%%%%%%%%%%%%%%%%%%%%

%\begin{appendix}

%\setcounter{equation}{0}
%\renewcommand{\theequation}{C\arabic{equation}}
%\setcounter{figure}{0}
%\renewcommand{\thefigure}{C\arabic{figure}}

{\bf \Large A summary of the XML tags and attributes}\label{sc:xtdb}

A complete and maybe updated version is available at
https://github/sundmanbo/XTDB.  Tagnames are {\bf bold} and names of
attributes are in {\em italics}.  Attributes which are mandatory are
indicated by a !M to the left.  Some explanations at the end.

\begin{table}[!h]
  \caption{Global tags}\label{tab:global}
  \begin{tabular}{|p{0.06\textwidth} p{0.15\textwidth} p{0.8\textwidth}|}\hline
    Tag     & Attributes & Note\\\hline

  XTDB    &         & Containing XML tags for an XTDB database.\\
!M        & Version & Version of XTDB used for this database.\\
!M        &Software & Name of software generating the database.\\
!M        &Date     & Year/month/day the database was written or last edited\\
!M        &Signature & person or organisation generating the database.\\\hline
  
  Defaults & & Optional tag to provide default values of attributes in
               different XML tags and some other things.  \\
           & LowT & Default value of low $T$ limit.\\
           & HighT & Default value of high $T$ limit.\\
         & Elements & For example ``VA'' (vacancy) and ``/-'' (the electron).\\
           & GlobalModel & Any model applicable to the whole database.\\\hline

  DatabaseInfo & & Optional tag with information about the database\\
           & Info & Free text (excluding the characters $< >$ \& ).\\
           & Date & Last update of the database information.\\\hline

  AppendXTDB & & Optional tag with additional files for the XTDB database.  
              The files should only contain XTBD tags.\\
         & Models & The {\bf ModelDescriptions} tag.\\
         & Parameters & With mainly {\bf Parameter} tags.\\
         & TPfuns & With some or all {\bf TPfun} tags.\\
         & Bibliography & With bibliographic tags.\\
         & Misc    & Whatever the database manager needs.\\\hline
  \end{tabular}
\end{table}

% Does not work to provide additional information
%\begin{itemize}
%\item A large XTDB file can use this feature to
%  separate parts of the content on other XTDB files.  All of them
%  should be specified here using this tag.
%\item\label{sc:phymod} The Gibbs energy of a phase may hace
%  contributions from magnetic and other properties modeled using
%  additional composition dependent parameters.  See the XTDB
%  documentation for more details.
%\end{itemize}

%%%%%%%%%%%%%%%%%%%%%%%%%%%%%%%%%%%%%%%%%%%%%%%%%%%%%%%%%%%%%%%%%%%%%%

\begin{table}[!h]
  \caption{The Element and Species tags}\label{sc:xtdbelements}
\begin{tabular}{|p{0.06\textwidth} p{0.15\textwidth} p{0.8\textwidth}|}\hline
  Tag & Attributes & Explanation\\\hline

  Element & & Specifies a chemical element in the database.  In addition
              the vacancy, denoted ``VA'', and the electron, denoted ``/-'',
              are included to handle defects and ions.\\
!M          & Id        &  Chemical element symbol, one or two letters, 
                         for example FE, H.  The Id is case insensitive.\\
%                         see section~\ref{sc:lettercase}.\\
          & Refstate  &  Name of the reference phase for the element.\\
!M          & Mass      &  Mass in g/mol\\
          & H298      &  Enthalpy difference between 0 and 298.15~K 
                         in the reference state.\\
          & S298      &  Entropy difference between 0 and 298.15~K in 
                         the reference state.\\\hline

  Species & & Specifies a chemical species used as a constituent of phases.  
                        The elements, except the electron, are also species.\\
!M          & Id        & Species name. Special rules apply. \\
                        % see section~\ref{sc:lettercase}.\\
!M          & Stoichiometry & One or more element {\em Id} each followed by 
                        an unsigned real or two integers separated by a ``/'' 
                        representing the stoichiometric ratio.
                        A ``/-'' or ``/+'' followed by a digit means a negative
                      or positive electric charge.  If no digit 1 is assumed.\\
%                        See section~\ref{sc:lettercase}.\\
                      & MQMQA     & For a constituent in the MQMQA model.\\
%                      See section~\ref{sc:MQMQA}.\\
                  & UNIQUAC   & For a constituent in the UNIQUAC model.\\\hline
%                      See section~\ref{sc:UNIQUAC}.\\\hline

\end{tabular}
\end{table}


%%%%%%%%%%%%%%%%%%%%%%%%%%%%%%%%%%%%%%%%%%%%%%%%%%%%%%%%%%%%%%%%%%%%%%

\begin{table}[!h]
  \caption{The phase tag.  All data belong to a phase.}
\begin{tabular}{|p{0.04\textwidth} p{0.18\textwidth} p{0.78\textwidth}|}\hline
  Tag & Attributes & Explanation\\\hline

  Phase & & All thermodynamic data is part of a phase.\\
!M       & Id & Phase name, special rules apply.\\
%  see sections~\ref{sc:lettercase} and \ref{sc:phaseID}. \\

!M       & Configuration &  Model for the configurational entropy. \\
%               see section~\ref{sc:config}.\\

         & State & Needed if EEC is used.\\\hline
%            see section~\ref{sc:eec}.\\\hline

CrystalStructure & & Optional inside a {\bf Phase} tag.\\
               & Prototype & Prototype phase \\
               & StrukturBericht & For example A3, B2, C14, D0\_3 etc.\\
               & PearsonSymbol & For example hR21 cI2.\\
               & SpaceGroup & For example 127, 166.\\\hline

  Sublattices & & Only once inside a {\bf Phase} tag.\\
!M        & NumberOf &  Number of sublattices, an integer value $>0$. \\
!M        & Multiplicities & One real value$>0$ for each sublattice. \\\hline
%               See~\ref{sc:alcexample}\\\hline

  Constituents & & Only inside the {\bf Sublattices} tag.\\
             & Sublattice & Can be omitted if only one sublattice.\\

             & WyckoffPosition & Optinonal specification.\\

  !M           & List &  Species {\em Id} separated by a space.\\\hline
% see section~\ref{sc:constituents}.\\\hline

  AmendPhase & & Optional tag inside a {\bf Phase} tag to specify a contribution
               from a physical model.\\
% see section~\ref{sc:phymod}, \ref{sc:modeltags} and \ref{sc:alcexample}.\\
     & Models & One or more model {\em Id}s, separated by a space. \\

           & & There can be an {\bf DisorderedPart} tag inside this tag.\\\hline

\end{tabular}
\end{table}

%%%%%%%%%%%%%%%%%%%%%%%%%%%%%%%%%%%%%%%%%%%%%%%%%%%%%%%%%%%%%%%%%%%%%%

\begin{table}[!h]
  \caption{The function tag.  Use the {\bf Trange} tag if several $T$ ranges, no provision for breakpoints in $P$.}

\begin{tabular}{|p{0.06\textwidth} p{0.10\textwidth} p{0.8\textwidth}|}\hline
  Tag & Attributes & Explanation\\\hline

  TPfun & & Defines a $T, P$ expression to be used in parameters or other functions.\\
  
!M        & Id & The {\em Id} can be used in the
                 {\em Expr} attribute of other functions or parameters. \\
%                see section~\ref{sc:lettercase}.  \\

        & LowT & Can be omitted if the default low $T$ limit applies.\\
                 !M       & Expr &  Simple mathematical expression, possibly terminated by ``;''.\\
% See section~\ref{sc:tpexpr}.  Use the {\bf Trange} tag if several ranges.\\
           
        & HighT & Can be omitted the default high $T$ limit applies.\\\hline

  Trange & & Only inside a {\bf TPfun} or {\bf Parameter} tag for an
             expression with several $T$ ranges.  The function and
             its first and second derivative must be continous.\\

!M & Expr & Simple mathematical expression possibly terminated by ``;''.\\
%             See section~\ref{sc:tpexpr}.\\

   & HighT & Can be omitted if the default high $T$ limit applies.\\\hline
\end{tabular}

\end{table}

%%%%%%%%%%%%%%%%%%%%%%%%%%%%%%%%%%%%%%%%%%%%%%%%%%%%%%%%%%%%%%%%%%%%%%

\begin{table}[!h]
  \caption{The parameter tag.  These tags the central part of an XTDB
    database.  The very compact form used in TDB files is
    retained.}\label{sc:partag}

\begin{tabular}{|p{0.06\textwidth} p{0.15\textwidth} p{0.8\textwidth}|}\hline
  Tag & Attributes & Explanation\\\hline

  Parameter & & Specifies the $T, P$ expression of a model parameter for a set of constituents of a phase.\\
  !M      & Id & As in a TDB file. \\
% See section~\ref{sc:param}\\

      & LowT & Can be omitted if the default low $T$ limit applies.\\
!M      & Expr & Simple mathematical expression possibly terminated by ``;''.
      If several ranges use a {\bf Trange} tag.  \\
% See section~\ref{sc:tpexpr}.\\
      & HighT & Can be omitted if the default high $T$ limit applies.\\
!M      & Bibref & Bibliographic reference.\\\hline
\end{tabular}
\end{table}

%%%%%%%%%%%%%%%%%%%%%%%%%%%%%%%%%%%%%%%%%%%%%%%%%%%%%%%%%%%%%%%%%%%%%%

\begin{table}[!h]
  \caption{The bibliography for parameters}
\begin{tabular}{|p{0.06\textwidth} p{0.15\textwidth} p{0.8\textwidth}|}\hline
  Tag & Attributes &  Explanation\\\hline

  Bibliography & & Contains bibliographic references.
                   There are no attributes.\\\hline

  Bibitem & & Only inside a {\bf Bibliography} tag.\\
!M      & Id &   Used as value in the {\em bibref} attribute for a parameter
            or model, normally a paper or a comment by the database manager.\\
!M      & Text & Reference to a paper or comment.\\
       & DOI & DOI of paper where the parameter was assessed.\\\hline
\end{tabular}
\end{table}

%%%%%%%%%%%%%%%%%%%%%%%%%%%%%%%%%%%%%%%%%%%%%%%%%%%%%%%%%%%%%%%%%%%%%%

\begin{table}[!h]
  \caption{The tags for current TDB models}\label{sc:models1}
\begin{tabular}{|p{0.06\textwidth} p{0.15\textwidth} p{0.8\textwidth}|}\hline
  Tag & Attributes & Explanation\\\hline

  ModelDescriptions & & Contains model tags usually with an {\em Id}
           attribute used in {\bf AmendPhase} tags inside a {\bf Phase} tag.
           Most models have one or more model parameter identifiers (MPID).\\
!M       & Software & Name of software using these models.  \\\hline

  Magnetic & & There are several magnetic models. \\
%           See section~\ref{sc:magmod}.\\
!M      & Id & This is used in {\em Models} attribute of the 
               {\bf AmendPhase} tag.  There are 3 variants:
               IHJBCC, IHJREST and IHJQX,\\
      & Aff   & The antiferromagnetic factor (-1, -3 or 0).\\

!M      & MPID1 & Specifies the Bohr magneton number MPID \\
!M      & MPID2 & Specifies a Curie or combined Curie/Neel temperature MPID\\
        & MPID3 & Specifies a Neel temperature MPID for IHJQX\\
!M      & Bibref & Where the model is described.\\\hline

 Permutations & & For FCC, HCP and BCC lattices a 4 sublattice tetrahedron
                     model. \\
% See section~\ref{sc:permut}.\\
!M     & Id & This is used in the {\em Models} attribute of the
                {\bf AmendPhase} tag.  Its can be either FCC4PERM or BCC4PERM.\\
!M     & Bibref & Where the model is explained.\\\hline

DisorderedPart & & Optional tag inside the {\bf AmendPhase} tag of an 
              ordered phase with or without order/disorder. \\
%              See section~\ref{sc:dispart}.\\

  & Disordered & Optional attribute with the {\em Id} of the disordered phase.\\

!M & Sum  & Number of sublattices in the ordered phase. \\
%  in eq.~\ref{eq:aver}.\\

   & Subtract & Must be ``Y'' if ordered part is complete.\\
% if eq.~\ref{eq:sub} in section~\ref{sc:dispart} should be used.\\
!M & Bibref & Where the model is described.\\\hline
\end{tabular}
\end{table}

%%%%%%%%%%%%%%%%%%%%%%%%%%%%%%%%%%%%%%%%%%%%%%%%%%%%%%%%%%%%%%%%%%%%%%

\begin{table}[!h]
  \caption{The tags for the new unary database}
\begin{tabular}{|p{0.06\textwidth} p{0.15\textwidth} p{0.8\textwidth}|}\hline
  Tag & Attributes & Explanation\\\hline

  Einstein & & The low $T$ vibrational model.\\
% see section~\ref{sc:einstein}.\\
!M      & Id & This Id is used in {\bf AmendPhase} tag.\\
!M      & MPID1 & Specifies the MPID for the Einstein $\theta$.\\
!M      & Bibref & Where the model is described.\\\hline

  Liquid2state & & The liquid 2-state model.\\
% see section~\ref{sc:liquid2state}.\\
!M      & Id & This Id is used in {\bf AmendPhase} tag.\\
!M      & MPID1 & Specifies the MPID for the 2-state transition energy. \\
!M      & MPID2 & Specifies the MPID for the Einstein $\theta$
                   for the low $T$ extrapolation.\\
!M      & Bibref & Where the model is described.\\\hline

  EEC & & Specifies that the Equi-entropy model applies to the whole database.\\
%          See section~\ref{sc:eec}.  The liquid {\bf Phase} tag must 
%          also have the {\em State} attribute equal to L. \\
!M      & Id     & has the value EEC.\\
!M      & Bibref & Where the model is described.\\\hline

\end{tabular}
\end{table}

%%%%%%%%%%%%%%%%%%%%%%%%%%%%%%%%%%%%%%%%%%%%%%%%%%%%%%%%%%%%%%%%%%%%%%

\begin{table}[!h]
  \caption{Miscellaneous tags}
\begin{tabular}{|p{0.06\textwidth} p{0.15\textwidth} p{0.8\textwidth}|}\hline
  Tag & Attributes & Explanation\\\hline

  TernaryXpol & & The extrapolation method for a ternary.\\
%                  See section~\ref{sc:terxpol}.\\
!M      & Phase & The {\em Id} of a phase.\\
!M      & Constituents & The {\em Id}s of 3 {\bf Species}  that are 
           constituents of the phase.\\
!M      & Xpol & The type of extrapolations, for example KKK if
           the Kohler method is used for all 3 binaries. \\
!M      & Bibref & Where the model is described.\\\hline

BinarySystem & & Optional tag for a database manager to surround
                a set of model parameters for a binary system.
                It can be used to list which assessed systems
                that are present in the database. \\
!M & System & The {\em Id}s of the two {\bf Element}s inside this tag.\\
   & CalcDia & Software dependent way to calculate the binary system\\\hline

\end{tabular}
\end{table}

\newpage

\begin{table}[!h]
  \caption{Summary of all tags}
%\begin{tabular}{|p{0.06\textwidth} p{0.15\textwidth} p{0.8\textwidth}|}\hline
\begin{tabular}{|p{0.21\textwidth} p{0.8\textwidth}|}\hline
  Tag        & Use \\\hline
  XTDB       &  Must start an XTDB file.\\
  Defaults   &  Optional default values.\\
  DatabaseInfo  & Optional database information.\\
  AppendXTDB &  Optional extra XTDB files for a large database.\\
  Element    &  Element data.\\
  Species    &  Species data.  Species are constituents of phases.\\
  Phase      &  Phase data, constituents and models.\\
  CrystalStructure  & Optional inside a {\bf Phase} tag.\\
  Sublattices & Only once inside a {\bf Phase} tag.\\
  Constituents  & Only inside a {\bf Sublattices} tag. \\ %10
  AmendPhase &  Only once inside a {\bf Phase} tag.\\
  TPfun      &  Specifies Id and expression. \\
  Trange     &  Inside a {\bf TPfun} tag.\\
  Parameter  &  Parameter identification and expression. \\
  Bibliography  & Surrounds parameter bibliograpy. \\
  Bibitem       & Bibliographic reference for a {\bf Parameter}.\\
  ModelDescriptions  & Surrounds model definition tags.\\
  Magnetic    & Specifies Id and MPID for a magnetic model for a phase.\\
  Permutation & Specifies Id for a permutation model for a phase.\\
  DisorderedPart & Optional specification of a phase with a disordred part.\\%20
  Einstein      & Specifies Id and MPID for the Einstein model.\\
  Liquid2State  & Specifies Id and MPID for the Liquid 2-state model.\\
  EEC           & Specifies Id for the Equal Entropi Criterium model.\\
  TernaryXpol   & Specifies a extrapolation model for a ternary.\\
  BinarySystem  & Optional specification for an assessed binary.\\\hline %25
  \end{tabular}
\end{table}

%\end{appendix}

\end{document}

